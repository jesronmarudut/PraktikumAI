\chapter{Mengenal Kecerdasan Buatan dan Scikit-Learn}
Buku umum yang digunakan adalah \cite{russell2016artificial} dan  
untuk sebelum UTS menggunakan buku \textit{Python Artificial Intelligence Projects for Beginners}\cite{eckroth2018python}.
Dengan praktek menggunakan python 3 dan editor anaconda dan library python scikit-learn.
Tujuan pembelajaran pada pertemuan pertama antara lain:
\begin{enumerate}
\item
Mengerti definisi kecerdasan buatan, sejarah kecerdasan buatan, perkembangan dan penggunaan di perusahaan
\item
Memahami cara instalasi dan pemakaian sci-kit learn
\item
Memahami cara penggunaan variabel explorer di spyder
\end{enumerate}
Tugas dengan cara dikumpulkan dengan pull request ke github dengan menggunakan latex pada repo yang dibuat oleh asisten riset.

\section{Teori}
Praktek teori penunjang yang dikerjakan :
\begin{enumerate}
\item
Buat Resume Definisi, Sejarah dan perkembangan Kecerdasan Buatan, dengan bahasa yang mudah dipahami dan dimengerti. Buatan sendiri bebas plagiat[hari ke 1](10)
\item
Buat Resume mengenai definisi supervised learning, klasifikasi, regresi dan unsupervised learning. Data set, training set dan testing set.[hari ke 1](10)
\end{enumerate}

\section{Instalasi}
Membuka https://scikit-learn.org/stable/tutorial/basic/tutorial.html. Dengan menggunakan bahasa yang mudah dimengerti dan bebas plagiat. 
Dan wajib skrinsut dari komputer sendiri.
\begin{enumerate}
\item
Instalasi library scikit dari anaconda, mencoba kompilasi dan uji coba ambil contoh kode dan lihat variabel explorer[hari ke 1](10)
\item
Mencoba Loading an example dataset, menjelaskan maksud dari tulisan tersebut dan mengartikan per baris[hari ke 1](10)
\item
Mencoba Learning and predicting, menjelaskan maksud dari tulisan tersebut dan mengartikan per baris[hari ke 2](10)
\item
mencoba Model persistence, menjelaskan maksud dari tulisan tersebut dan mengartikan per baris[hari ke 2](10)
\item 
Mencoba Conventions, menjelaskan maksud dari tulisan tersebut dan mengartikan per baris[hari ke 2](10)
\end{enumerate}


\section{Penanganan Error}
Dari percobaan yang dilakukan di atas, apabila mendapatkan error maka:

\begin{enumerate}
	\item
	skrinsut error[hari ke 2](10)
	\item
Tuliskan kode eror dan jenis errornya [hari ke 2](10)
	\item
Solusi pemecahan masalah error tersebut[hari ke 2](10)

\end{enumerate}

<<<<<<< HEAD
\section{Teori/Mhd Zulfikar Akram Nasution/1164081}
\begin{enumerate}
\item Definisi, Sejarah dan Perkembangan Kecerdasan Buatan
\begin{itemize}
\item Definisi
\par
Kecerdasan Buatan adalah kecerdasan yang ditambahkan kepada suatu sistem yang bisa diatur dalam konteks ilmiah yang berhubungan dengan pemanfaatan mesin untuk memecahkan persoalan yang rumit dengan cara yang lebih manusiawi. 
\par
\item Sejarah dan Perkembangan
\par
Sejarah dan perkembangan kecerdasan buatan terjadi pada musim panas tahun 1956 tercatat adanya seminar mengenai AI di Darmouth College. Seminar pada waktu itu dihadiri oleh sejumlah pakar komputer dan membahas potensi komputer dalam meniru kepandaian manusia. Akan tetapi perkembangan yang sering terjadi semenjak diciptakannya LISP, yaitu bahasa kecerdasan buatan yang dibuat tahun 1960 oleh John McCarthy. Istilah pada kecerdasan buatan atau Artificial Intelligence diambil dari Marvin Minsky dari MIT. Dia menulis karya ilmiah berjudul Step towards Artificial Intelligence,The Institute of radio Engineers Proceedings 49, January 1961.
\end{itemize}

\item Definisi Supervised Learning, Unsupervised Learning, Klasifikasi, Regresi, Data Set, Training Set dan Testing Set
\begin{itemize}
\item Supervised Learning dan Unsupervised Learning
\par
Supervised learning merupakan sebuah pendekatan dimana sudah terdapat data yang dilatih, dan terdapat variable yang ditargetkan sehingga tujuan dari pendekatan ini adalah mengkelompokan suatu data ke data yang sudah ada. Sedangkan unsupervised learning tidak memiliki data latih, sehingga dari data yang ada, kita mengelompokan data tersebut menjadi 2 bagian atau 3 bagian dan seterusnya.
\item Klasifikasi
\par
Klasifikasi adalah salah satu topik utama dalam data mining atau machine learning. Klasifikasi yaitu suatu pengelompokan data dimana data yang digunakan tersebut mempunyai kelas label atau target.
\item Regresi
\par
Regresi adalah Supervised learning tidak hanya mempelajari classifier, tetapi juga mempelajari fungsi yang dapat memprediksi suatu nilai numerik. Contoh, ketika diberi foto seseorang, kita ingin memprediksi umur, tinggi, dan berat orang yang ada pada foto tersebut.
\item Data Set
\par
Data set adalah cabang aplikasi dari Artificial Intelligence/Kecerdasan Buatan yang fokus pada pengembangan sebuah sistem yang mampu belajar sendiri tanpa harus berulang kali di program oleh manusia.
\item Training Set
\par
Training set yaitu jika pasangan objek, dan kelas yang menunjuk pada objek tersebut adalah suatu contoh yang telah diberi label akan menghasilkan suatu algoritma pembelajaran.
\item Testing Set
\par
Testing set digunakan untuk mengukur sejauh mana classifier berhasil melakukan klasifikasi dengan benar.
\end{itemize}
\item Instalasi Scikit-Learn dari Anaconda
\begin{itemize}
\item Pertama install Anaconda di pc masing-masing
\item Kemudian buka cmd untuk menginstall scikit-learn
\item Ketik perintah "conda install scikit-learn" dan pilih "y"
\end{itemize}
\begin{figure}[ht]
\centering
\includegraphics[scale=0.6]{figures/1.png}
\caption{Install Scikit-Learn Conda}
\label{Proses Instalasi}
\end{figure}
\begin{itemize}
\item Lalu ketik "pip install -U scikit-learn" untuk memasukkan anaconda ke python
\end{itemize}
\begin{figure}[ht]
\centering
\includegraphics[scale=0.5]{figures/2.png}
\caption{Install Scikit-Learn ke Python}
\label{Gabung Conda dan Python}
\end{figure}
\begin{itemize}
\item Setelah itu, kompilasi kode di dalam python dengan ketik "python", lalu "print('Zulfikar')" maka akan menghasilkan seperti gambar berikut.
\end{itemize}
\begin{figure}[ht]
\centering
\includegraphics[scale=0.6 ]{figures/3.png}
\caption{Kompilasi Kode}
\label{Kompilasi Kode}
\end{figure}
\item Loading an Example Dataset
\begin{itemize}
\item Ketik perintah berikut "from sklearn import datasets" untuk mengimport dataset dari sklearn.
\end{itemize}
\begin{figure}[ht]
\centering
\includegraphics[scale=0.5]{figures/4.png}
\caption{Import Datasets}
\label{Import Datasets}
\end{figure}
\begin{itemize}
\item Kemudian ketik perintah berikut  untuk membuat variable iris yang berisi datasets.
\end{itemize}
\begin{figure}[ht]
\centering
\includegraphics[scale=0.9]{figures/5.png}
\caption{Buat variable iris}
\label{Variable Iris}
\end{figure}
\begin{itemize}
\item Lalu ketik perintah berikut untuk membuat variable digits yang berisi datasets, dan juga untuk melihat isi data dari datasets seperti gambar 1.6 .
\end{itemize}
\begin{figure}[ht]
\centering
\includegraphics[scale=0.7]{figures/8.png}
\caption{Buat variable digits}
\label{Variable Digits}
\end{figure}
\end{enumerate}

\section{Jesron Marudut Hatuan/1164077}
\subsection{Teori}
\begin{enumerate}
\item Definisi, sejarah, dan perkembangan kecerdasan buatan.
\subitem Kecerdasan Buatan (Artificial Intelligence atau AI) dapat didefinisikan sebagai kecerdasan yang ditunjukkan oleh suatu entitas buatan. Sistem seperti ini biasanya dianggap komputer. Kecerdasan diciptakan lalu dimasukkan ke dalam suatu mesin atau komputer supaya dapat melakukan pekerjaan-pekerjan yang dapat dilakukan manusia.
\subitem Sebenarnya area Kecerdasan Buatan (Artificial Intelligence) atau disingkat dengan AI, dimulai dari munculanya komputer sekitar tahun 1940-an, meskipun sejarah perkembangannya dapat dilacak dari zaman Mesir kuno. Pada akhir tahun 1955, Newell dan Simon mengembangkan The Logic Theorist atau program AI terdahulu. Program ini merepresentasikan masalah sebagai model pohon, lalu penyelesaiannya dengan  memilih cabang yang akan menghasilkan kesimpulan terbenar. Program tersebut berdampak besar dan menjadi batu loncatan dalam mengembangkan bidang AI. Pada tahun 1956 John McCarthy dari  Massacuhetts Institute of Technology dianggap sebagai bapak AI, menyelenggarakan konferensi untuk menarik para ahli komputer bertemu, dengan  nama kegiatan The Dartmouth Summer Research Project On AI. Konferensi Dartmouth saat itu mempertemukan para pendiri dalam AI, dan bertugas untuk meletakkan dasar bagi masa depan  pemgembangan dan penelitian AI. John McCarthy  disaat itu mengusulkan definisi AI adalah AI merupakan cabang dari ilmu komputer yang berfokus pada pengembangan komputer agar mempunyai kemampuan dan berprilaku seperti manusia.
\item  Definisi supervised learning, klasifikasi, regresi, dan unsupervised learning. Data set, training set dan testing set. 
\subitem Supervised learning merupakan sebuah pendekatan dimana sudah terdapat data yang dilatih, dan terdapat variable yang ditargetkan sehingga tujuan dari pendekatan ini adalah mengkelompokan suatu data ke data yang sudah ada. Sedangkan unsupervised learning tidak memiliki data latih, sehingga dari data yang ada, kita mengelompokan data tersebut menjadi 2 bagian atau 3 bagian dan seterusnya.
\subitem Klasifikasi adalah salah satu topik utama dalam data mining atau machine learning. Klasifikasi yaitu suatu pengelompokan data dimana data yang digunakan tersebut mempunyai kelas label atau target.
\subitem Regresi adalah Supervised learning tidak hanya mempelajari classifier, tetapi juga mempelajari fungsi yang dapat memprediksi suatu nilai numerik. Contoh, ketika diberi foto seseorang, kita ingin memprediksi umur, tinggi, dan berat orang yang ada pada foto tersebut.
\subitem Data set adalah cabang aplikasi dari Artificial Intelligence/Kecerdasan Buatan yang fokus pada pengembangan sebuah sistem yang mampu belajar sendiri tanpa harus berulang kali di program oleh manusia.
\subitem Training set yaitu jika pasangan objek, dan kelas yang menunjuk pada objek tersebut adalah suatu contoh yang telah diberi label akan menghasilkan suatu algoritma pembelajaran.
\subitem Testing set digunakan untuk mengukur sejauh mana classifier berhasil melakukan klasifikasi dengan benar\cite{zhu2009introduction}.
\end{enumerate}


\subsection{Instalasi}
\subsubsection{Instalasi Library Scikit dari Anaconda}
\begin{enumerate}
\item Sediakan aplikasi Anaconda terlebih dahulu
\begin{figure}[ht]
\centerline{\includegraphics[width=1\textwidth]{figures/0.PNG}}
\caption{Applikasi Anaconda.}
\end{figure}
\item Setelah di install, masukkan script dibawah ini untuk melihat versi Python dan Anacondanya
\begin{figure}[ht]
\centerline{\includegraphics[width=0.75\textwidth]{figures/1.JPEG}}
\caption{Versi Anaconda.}
\end{figure}
\item  Selanjutnya masukkan perintah 'pip install -U scikit-learn'
\begin{figure}[ht]
\centerline{\includegraphics[width=0.75\textwidth]{figures/2.JPEG}}
\caption{Instalasi.}
\end{figure}
\item  Selanjutnya masukkan perintah 'conda install  scikit-learn'
\begin{figure}[ht]
\centerline{\includegraphics[width=0.75\textwidth]{figures/3.JPEG}}
\caption{Langkah installasi anaconda.}
\end{figure}
\item  Selanjutnya masukkan perintah 'python' dan 'print ('jesron')
\begin{figure}[ht]
\centerline{\includegraphics[width=0.5\textwidth]{figures/4.JPEG}}
\caption{Langkah terakhir.}
\end{figure}
\end{enumerate}

\section{Instalasi/Mhd Zulfikar Akram Nasution/1164081}
\begin{enumerate}
\item Menjelaskan Kode dari Learning and Predicting
\begin{itemize}
\item Pertama import file smv dari sklearn seperti pada gambar 1.12
\end{itemize}
\begin{figure}[ht]
\centering
\includegraphics[scale=0.9]{figures/2_1.png}
\caption{Import file svm}
\label{Import svm}
\end{figure}
\begin{itemize}
\item Kemudian buat variabel clf seperti pada gambar 1.13
\end{itemize}
\begin{figure}[ht]
\centering
\includegraphics[scale=0.9]{figures/2_2.png}
\caption{Buat variable Classifier}
\label{Variabel clf}
\end{figure}
\begin{itemize}
\item Lalu ketik kode berikut untuk meliat array baru dari syntax python [:-1] sepert padai gambar 1.14
\end{itemize}
\begin{figure}[ht]
\centering
\includegraphics[scale=0.7]{figures/2_3.png}
\caption{Lihat array baru dengan syntac Python}
\label{Syntax python}
\end{figure}
\begin{itemize}
\item Selanjutnya ketikkan kode berikut untuk melihat penggolongan array seperti pada gambar 1.15
\end{itemize}
\begin{figure}[ht]
\centering
\includegraphics[scale=0.7]{figures/2_4.png}
\caption{Lihat classifier array}
\label{Classifier Array}
\end{figure}
\item Model Persistence
\begin{itemize}
\item Pertama Import dulu file dari sklearn
\end{itemize}
\begin{figure}[ht]
\centering
\includegraphics[scale=0.7]{figures/2_5.png}
\caption{Import file}
\end{figure}
\begin{itemize}
\item Kemudian buat variable classifier dengan gamma=scale
\end{itemize}
\begin{figure}[ht]
\centering
\includegraphics[scale=0.9]{figures/2_6.png}
\caption{Variable classifier}
\end{figure}
\begin{itemize}
\item Lalu buat variable iris dan (X,y)
\end{itemize}
\begin{figure}[ht]
\centering\includegraphics[scale=0.9]{figures/2_7.png}
\caption{Variable iris}
\end{figure}
\begin{itemize}
\item Selanjutnya kita akan melihat penyesuaian classifier
\end{itemize}
\begin{figure}[ht]
\centering
\includegraphics[scale=0.7]{figures/2_8.png}
\caption{Penyesuaian Classifier}
\end{figure}
\begin{itemize}
\item Kemudian import pickle untuk melihat hasil array dan hasil y
\end{itemize}
\begin{figure}[ht]
\centering
\includegraphics[scale=0.7]{figures/2_9.png}
\caption{Import Pickle}
\end{figure}
\item Conventions
\begin{itemize}
\item Pertama import numpy menjadi np serta import random projection
\end{itemize}
\begin{figure}[ht]
\centering
\includegraphics[scale=0.7]{figures/2_10.png}
\caption{Import numpy}
\end{figure}
\begin{itemize}
\item Kemudian buat variable rng dengan type random
\end{itemize}
\begin{figure}[ht]
\centering
\includegraphics[scale=0.7]{figures/2_11.png}
\caption{Variable rng}
\end{figure}
\begin{itemize}
\item Lalu buat variable X, dan lihat hasil rng random yang keluar
\end{itemize}
\begin{figure}[ht]
\centering
\includegraphics[scale=0.9]{figures/2_12.png}
\caption{Variable X dan hasil random}
\end{figure}
\begin{itemize}
\item Setelah itu buat variable transformer dengan type random
\end{itemize}
\begin{figure}[ht]
\centering
\includegraphics[scale=0.9]{figures/2_13.png}
\caption{Variable transformer random}
\end{figure}
\begin{itemize}
\item Berikutnya itu buat variable X new dengan type yang ada pada tranformer
\end{itemize}
\begin{figure}[ht]
\centering
\includegraphics[scale=0.9]{figures/2_14.png}
\caption{Variable X new type pada transformer}
\end{figure}
\begin{itemize}
\item Kemudian lihat hasil dari X new
\end{itemize}
\begin{figure}[ht]
\centering
\includegraphics[scale=0.9]{figures/2_15.png}
\caption{Hasil dari X new type pada transformer}
\end{figure}
\item Screenshoot Error pada gambar 1.27
\begin{figure}[ht]
\centering
\includegraphics[scale=0.7]{figures/2_16.png}
\caption{Screenshoot Error}
\end{figure}
\item Kode yang error yaitu "joblib" karena belum ada library nya
\item Solusi dari masalah yang error 

\end{enumerate}